% !TEX TS-program = xelatex
% !TEX encoding = UTF-8 Unicode
% !Mode:: "TeX:UTF-8"

\documentclass{resume}
\usepackage{zh_CN-Adobefonts_external} % Simplified Chinese Support using external fonts (./fonts/zh_CN-Adobe/)
%\usepackage{zh_CN-Adobefonts_internal} % Simplified Chinese Support using system fonts
\usepackage{linespacing_fix} % disable extra space before next section
\usepackage{cite}

\begin{document}
\pagenumbering{gobble} % suppress displaying page number

\name{谭琼}

\basicInfo{
  \email{mf1832143@smail.nju.edu.cn} \textperiodcentered\
  \phone{135-8517-8653} }

\section{\faGraduationCap\  教育背景}
\datedsubsection{\textbf{南京大学}, 南京}{2018 -- 至今}
\textit{女 / 1996 /在读硕士研究生}\ 软件工程, 预计 2020 年 6 月毕业
\datedsubsection{\textbf{南京大学}, 南京}{2014 -- 2018}
\textit{学士}\ 软件工程

\section{\faUsers\ 项目经历}

\datedsubsection{\textbf{基于K线型态的股票量化分析系统 Web项目}}{2018 年3月 -- 5月}
\role{HTML+CSS+JavaScript, Python}  {个人项目}
\begin{onehalfspacing}
\begin{itemize}
  \item 作为毕业设计,项目主要功能有筛选股票池、添加自选股、回测和策略设置,为投资者的选股提供建议,推荐优质股票,也供投资者自定义策略参数
  \item 项目前端采用BootStrap框架,后端采用Django框架
\end{itemize}
\end{onehalfspacing}

\datedsubsection{\textbf{看板项目 Web项目}}{2017 年6月 -- 8月}
\role{jsp+CSS+JavaScript, Java}  {华为南研所实习生}
\begin{onehalfspacing}
\begin{itemize}
  \item 该项目的目标用户为公司内部人员,用户可以在看板上跟踪管理当前项目的进度
  \item 项目前端采用AngularJS框架,后端使用Java实现
  \item 负责前端部分
\end{itemize}
\end{onehalfspacing}

\datedsubsection{\textbf{酒店管理系统 J2EE项目}}{2017 年3月}
\role{jsp+CSS+JavaScript, Java}  {个人项目}
\begin{onehalfspacing}
\begin{itemize}
  \item 作为课程作业,项目有三个角色:会员(注册、激活)、客栈(申请开店、发布计划)、经理(审核申请,会员结算,查看统计信息)
  \item 整体采用MVC架构,jsp实现view层,Servlet实现controller层。jsp通过AJAX异步提交请求到Servlet,Servlet负责将相应请求产生的数据带给jsp
\end{itemize}
\end{onehalfspacing}

\datedsubsection{\textbf{AnyQuant股票数据分析系统 Java桌面应用}}{2016 年4月 -- 6月}
\role{Java}  {团队项目}
\begin{onehalfspacing}
\begin{itemize}
  \item 作为课程作业,对股票数据进行分析、预测
  \item 项目采用三层架构,分为界面层、业务逻辑层和数据访问层
  \item 主要负责大部分的界面,使用了Java FX技术
\end{itemize}
\end{onehalfspacing}

\datedsubsection{\textbf{快递物流系统 Java桌面应用}}{2015 年10月 -- 12月}
\role{Java}  {团队项目}
\begin{onehalfspacing}
\begin{itemize}
  \item 作为课程作业,模拟了快递从产生订单到收货的整个流程,通信采用RMI技术
  \item 项目采用分层架构
  \item 主要负责数据层以及逻辑层的实现
\end{itemize}
\end{onehalfspacing}


% Reference Test
%\datedsubsection{\textbf{Paper Title\cite{zaharia2012resilient}}}{May. 2015}
%An xxx optimized for xxx\cite{verma2015large}
%\begin{itemize}
%  \item main contribution
%\end{itemize}

\section{\faCogs\ IT 技能}
% increase linespacing [parsep=0.5ex]
\begin{itemize}[parsep=0.5ex]
  \item 编程语言: 熟悉HTML, CSS, JavaScript, Java等
\end{itemize}

%% Reference
%\newpage
%\bibliographystyle{IEEETran}
%\bibliography{mycite}
\end{document}
